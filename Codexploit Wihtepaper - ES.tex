\documentclass[10pt, a4paper, oneside]{article}
\usepackage[hidelinks]{hyperref}
\usepackage{jucs2e}
\usepackage{graphicx}
\usepackage{url}
\usepackage{ulem}
\usepackage{mathtools}
\usepackage{scalerel}
\usepackage{setspace}
\usepackage[strict]{changepage}
\usepackage{caption}
\usepackage[letterspace=-50]{microtype}
%\usepackage{fontspec}
\usepackage{afterpage}
\usepackage{ragged2e}
%\setmainfont{Times New Roman}
\usepackage[T1]{fontenc}
\usepackage[utf8]{inputenc}
\usepackage{times} % O \usepackage{mathptmx}
\usepackage{titlesec}
\usepackage[spanish]{babel}
\usepackage[alf]{abntex2cite} % o [author-year] para otro estilo

\titleformat*{\section}{\Large\bfseries}
\titleformat*{\subsection}{\normalsize\bfseries}

\renewcommand{\baselinestretch}{0.9} 

\graphicspath{{./figures/}}

\usepackage[textwidth=8cm, margin=0cm, left=4.6cm, right=4.2cm, top=3.9cm, bottom=6.8cm, a4paper, headheight=0.5cm, headsep=0.5cm]{geometry}
\usepackage{fancyhdr}
\usepackage[format=plain, labelfont=it, textfont=it, justification=centering]{caption}
\usepackage{breakcites}
\usepackage{microtype}
 
\apptocmd{\frame}{}{\justifying}{}

\urlstyle{same}
\pagestyle{fancy}

\newcommand\jucs{Codexplo.it Project Whitepaper}
\newcommand\jucsvol{Versión 0.01}
\newcommand\jucspages{páginas: 1--7} % ajustar según número de páginas
\renewcommand\jucssubmitted{19 jun. 2025} % fecha actual de compilación
\newcommand\jucsaccepted{26 jun. 2025} % si no hay aceptación formal, dejar vacío o "N/A"
\newcommand\jucsappeared{5 jul. 2025} % fecha de publicación o última revisión
\newcommand\jucslicence{, Licencia: MIT License}
\newcommand\startingPage{1}
\setcounter{page}{\startingPage}

% Autor
\newcommand\paperauthor{{Vilela C.: }}

% Título
\newcommand\papertitle{Whitepaper \\ Proyecto Codexplo.it}

% íconos
\usepackage{fontawesome}
\usepackage{hyperref}

% Contenido del encabezado principal
\header{\paperauthor \papertitle}

\begin{document}

\title{{\fontsize{14pt}{14pt}\selectfont{\vspace*{-3mm}\papertitle\vspace*{-1mm}}}}

\author{%
  {\bfseries\fontsize{10pt}{12pt}\selectfont Carlos Vilela}\\
  {\fontsize{9pt}{12pt}\selectfont codexplo.it (Proyecto Personal)}\\
  {\faGithub\ \href{https://github.com/cod3xpl0it}{github.com/cod3xpl0it}}
  {\faGlobe\ \href{https://codexplo.it}{https://codexplo.it}}
}

\label{first}
\maketitle

{\fontfamily{ptm}\selectfont
\begin{abstract}
{\fontsize{9pt}{9pt}\selectfont{\vspace*{-2mm}
Este Whitepaper presenta Codexplo.it, un proyecto personal de ingeniería informática enfocado en el desarrollo de soluciones innovadoras de software y hardware mediante el aprendizaje práctico y la experimentación técnica, con el objetivo de explorar técnicas avanzadas de programación, sistemas embebidos y arquitecturas de computadoras, buscando mejorar habilidades profesionales, estimular la creatividad y construir aplicaciones útiles que puedan compartirse con la comunidad de código abierto, además de formar un portafolio diverso que demuestre competencias técnicas aplicadas en contextos reales.}}
\end{abstract}}

{\fontfamily{ptm}\selectfont
\begin{keywords}
{\fontsize{9pt}{9pt}\selectfont{
ingeniería informática, sistemas embebidos, desarrollo de software, programación avanzada, aprendizaje práctico, proyectos personales, experimentación tecnológica, código abierto}}
\end{keywords}}

\section{Introducción}

La ingeniería informática está en constante evolución, adoptando nuevas tecnologías y metodologías, y este Whitepaper presenta Codexplo.it, un proyecto personal que explora soluciones innovadoras en software y hardware mediante la experimentación práctica y el aprendizaje, con la finalidad de mejorar habilidades profesionales, estimular la creatividad y contribuir con proyectos útiles para la comunidad tecnológica, además, el documento describe la motivación, objetivos, metodologías y resultados esperados del proyecto.

\section{Motivación}

El proyecto Codexplo.it nació del deseo de consolidar el conocimiento adquirido a lo largo de los estudios académicos y las experiencias personales, transformando ese aprendizaje en algo práctico, creativo y visible mediante un portafolio personal, este espacio tiene la finalidad de exponer proyectos personales desarrollados con creatividad, libertad y experimentación, explorando diferentes áreas de la ingeniería informática, ya sea en software, hardware o en la integración de ambos.\newline

La idea principal es construir continuamente habilidades técnicas y personales, aplicando conceptos aprendidos formalmente e incorporando ideas provenientes del día a día que, incluso fuera del ámbito técnico, han contribuido a mi desarrollo personal y profesional.\newpage

\section{Definiciones del Proyecto}

\begin{itemize}
  \item \textbf{Nombre del proyecto:} Codexplo.it  
  \item \textbf{Sitio oficial:} \texttt{https://codexplo.it}  
  \item \textbf{Repositorio GitHub:} \texttt{https://github.com/cod3xpl0it}  
  \item \textbf{Origen del nombre:}  
  El nombre “codexplo.it” resulta de la combinación entre “code” (código) y “exploit” (explotar), expresando el propósito del proyecto de investigar y experimentar tecnologías en software y hardware.\newline
  
  El dominio “.it” es un \textit{ccTLD} (country code Top-Level Domain), que identifica a Italia en internet, incorporado intencionalmente para formar el nombre de manera creativa, haciéndolo más corto, moderno y fácil de memorizar.  
\end{itemize}

\section{Visión General del Proyecto}

El proyecto \textbf{Codexplo.it} es una iniciativa personal enfocada en integrar y organizar diversas experiencias y soluciones técnicas desarrolladas en las áreas de software y hardware, promoviendo el aprendizaje práctico y el desarrollo continuo de competencias en ingeniería informática, dirigido a entusiastas, estudiantes y profesionales que valoran la innovación y la aplicación real del conocimiento técnico, ofreciendo un portafolio accesible y transparente que ilustra la evolución de las habilidades adquiridas con el tiempo, mientras fomenta el intercambio de ideas y la colaboración dentro de la comunidad tecnológica.

\section{Objetivos}

El proyecto \textbf{Codexplo.it} tiene como objetivo principal servir como una plataforma personal de experimentación, aprendizaje y demostración de soluciones tecnológicas, reuniendo creaciones originales en un portafolio accesible, organizado y técnicamente relevante. Entre sus objetivos específicos se encuentran:

\begin{itemize}
  \item Aplicar, de forma práctica, el conocimiento adquirido tanto en la formación académica como en las experiencias personales, promoviendo un aprendizaje más profundo.
  \item Desarrollar y documentar proyectos originales, permitiendo una reflexión continua sobre procesos, decisiones técnicas y resultados alcanzados.
  \item Estimular la creatividad, el pensamiento crítico y la autonomía, mediante la libertad para explorar diversas tecnologías, metodologías y lenguajes de programación.
  \item Construir y mejorar habilidades técnicas e interpersonales, con enfoque en el crecimiento profesional y en la creación de una ventaja competitiva.
  \item Crear un repositorio público que pueda inspirar, colaborar y contribuir con la comunidad tecnológica.\newpage
\end{itemize}

\section{Metodología}

La metodología adoptada en el proyecto \textbf{Codexplo.it} se basa en un enfoque exploratorio y práctico, donde cada proyecto desarrollado funciona como una oportunidad de aprendizaje activo y desarrollo de habilidades, la idea central es aplicar los conocimientos adquiridos, explorar nuevas tecnologías, proponer soluciones creativas y reflexionar sobre los resultados, todo ello a través de un proceso continuo, iterativo y autónomo.

El proceso metodológico sigue una estructura flexible, compuesta por etapas que se adaptan según los desafíos a enfrentar:

\begin{itemize}
  \item \textbf{Identificación del problema o desafío}: definición del alcance, objetivos y posibles soluciones.
  \item \textbf{Investigación y fundamentación}: recopilación de referencias técnicas, buenas prácticas y tecnologías viables.
  \item \textbf{Desarrollo práctico}: implementación de la solución usando herramientas y lenguajes adecuados.
  \item \textbf{Documentación y reflexión}: registro del proceso, decisiones tomadas y lecciones aprendidas.
  \item \textbf{Compartir}: publicación de los resultados en el portafolio, incluyendo código fuente, documentación, descripciones y posibles mejoras futuras.
\end{itemize}

Cada desafío comienza con un problema, seguido por la investigación, creación de la solución y registro de todo el proceso, como códigos y aprendizajes, compartiendo la documentación en el portafolio, promoviendo el intercambio de conocimientos y la evolución continua.

\subsection{Historial}

La idea detrás del proyecto \textbf{Codexplo.it} surgió a principios de enero de 2025 como una iniciativa personal para consolidar el aprendizaje técnico y la experimentación práctica; con el tiempo, el concepto evolucionó, dando origen a un proyecto estructurado, con identidad y propósito claros.

\begin{table}[ht]
    \centering
    \setlength\tabcolsep{6pt}
    \begin{tabular}{ |l|l|p{6.8cm}| }
        \hline
        \bfseries Fecha & \bfseries Fase & \bfseries Descripción \\
        \hline
        06/01/2025 & Ideación & Inicio de la reflexión sobre la importancia de reunir proyectos personales en un portafolio técnico, enfatizando el aprendizaje teórico y práctico \\
        \hline
        Feb–Abr/2025 & Concepción & Desarrollo de la identidad del proyecto, definición del nombre, organización de la propuesta y elaboración de los planes estructurales iniciales \\
        \hline
        11/05/2025 & Registro & Adquisición oficial del dominio \texttt{codexplo.it}, marcando el inicio formal del proyecto como una plataforma pública y accesible \\
        \hline
        May/2025–Presente & Implementación & Desarrollo de contenido, organización de proyectos previos, aplicación de metodologías prácticas y construcción continua del portafolio \\
        \hline
    \end{tabular}
    \caption{\fontsize{10pt}{11pt}\selectfont{\itshape{Evolución del proyecto Codexploit desde su concepción hasta la fase actual de implementación}}}
    \label{table:history}
\end{table}
\newpage

\section{Requisitos del Proyecto}

Para que el proyecto \textbf{Codexploit} se desarrolle de manera coherente y alcance sus objetivos, es esencial definir requisitos claros que orienten sus actividades, organización y resultados esperados, estos requisitos se dividen en funcionales, no funcionales e inversos, abarcando tanto la ejecución práctica como los límites y directrices del proyecto.

\subsection{Requisitos Funcionales}

Los requisitos funcionales especifican las acciones y entregables que el proyecto debe proporcionar para cumplir el propósito de reunir y presentar las problemáticas, fomentar el aprendizaje y documentar el progreso técnico.

\begin{itemize}
    \item Organizar y presentar las problemáticas de forma clara, accesible y estructurada, facilitando el seguimiento del desarrollo técnico.
    \item Documentar el proceso de aprendizaje mediante la experimentación práctica, pruebas y validación continua.
    \item Compartir resultados, códigos y documentación que puedan contribuir con la comunidad tecnológica.
    \item Actualizar el portafolio con nuevas problemáticas y mejoras, reflejando la evolución de las habilidades a lo largo del tiempo.
    \item Mantener una comunicación transparente sobre los objetivos, metodologías y progreso del proyecto.
\end{itemize}

\subsection{Requisitos No Funcionales}

Los requisitos no funcionales establecen cualidades y restricciones que deben guiar el proyecto para garantizar su sostenibilidad, relevancia e impacto positivo.

\begin{itemize}
    \item Mantener un enfoque práctico y autodidacta, valorando la experimentación y el aprendizaje continuo.
    \item Asegurar consistencia y calidad en la documentación y el contenido presentado.
    \item Garantizar accesibilidad del portafolio para diferentes públicos, respetando estándares de usabilidad y claridad.
    \item Adoptar prácticas éticas, respetando derechos de autor y privacidad en la información compartida.
    \item Preservar la flexibilidad para adaptar e incorporar nuevas tecnologías o áreas de interés.
\end{itemize}

\subsection{Requisitos Inversos}

Los requisitos inversos definen limitaciones y comportamientos que deben evitarse en el proyecto para preservar el enfoque, calidad e integridad.

\begin{itemize}
    \item Evitar la dispersión del enfoque del proyecto con temas o contenidos no relacionados con la ingeniería informática y el desarrollo personal.
    \item Prevenir la publicación de información personal sensible o datos que puedan comprometer la privacidad.
    \item No sacrificar la calidad de la documentación en favor de la cantidad de proyectos presentados.
    \item Evitar prácticas que comprometan la originalidad o la ética, como plagio o uso no autorizado de materiales protegidos.
    \item Impedir enfoques que limiten el crecimiento o restrinjan la experimentación, innovación y creatividad.
\end{itemize}

\section{Tecnologías y Herramientas Utilizadas}

Las tecnologías utilizadas en los proyectos que componen \textbf{Codexplo.it} se eligen basándose en los objetivos de cada problemática, priorizando herramientas gratuitas, abiertas y sin restricciones, para garantizar fácil acceso, incentivar la experimentación y apoyar el aprendizaje, siempre buscando soluciones flexibles, bien documentadas y adecuadas a los desafíos presentados, manteniendo un equilibrio entre practicidad, innovación, desarrollo técnico y creatividad.

\section{Aspectos Legales y Licenciamiento}

El proyecto \textbf{Codexplo.it} usa la licencia MIT, que permite usar, modificar y compartir el contenido libremente, proporcionando la libertad para que todos contribuyan y utilicen el proyecto.

\section{Análisis SWOT}

A continuación, se presenta un análisis SWOT (Fortalezas, Debilidades, Oportunidades y Amenazas) del proyecto \textbf{Codexplo.it}, con enfoque en su desarrollo como iniciativa personal en el área de ingeniería informática.

\subsection*{Fortalezas (Strengths)}
\begin{itemize}
  \item Proyecto autónomo, con total libertad creativa.
  \item Enfoque en aprendizaje práctico y continuo.
  \item Documentación organizada y estructurada.
  \item Licencia MIT que permite colaboración abierta.
  \item Integración entre software y hardware.
\end{itemize}

\subsection*{Debilidades (Weaknesses)}
\begin{itemize}
  \item Depende exclusivamente del tiempo y motivación personal.
  \item Recursos limitados para proyectos más complejos.
\end{itemize}

\subsection*{Oportunidades (Opportunities)}
\begin{itemize}
  \item Compromiso con la comunidad open source.
  \item Posibilidad de transformar contenido en material educativo.
  \item Potencial para asociaciones y colaboraciones externas.
  \item Publicación en plataformas técnicas como GitHub.
\end{itemize}

\subsection*{Amenazas (Threats)}
\begin{itemize}
  \item Obsolescencia de tecnologías utilizadas.
  \item Competencia con proyectos más estructurados o financiados.
  \item Exposición a errores técnicos o fallas de seguridad.
\end{itemize}

\section{Conclusión y Próximos Pasos}

Este Whitepaper presentó el alcance, objetivos, metodología y estructura del proyecto \textbf{Codexploit}, demostrando su potencial como plataforma para aprendizaje práctico y desarrollo técnico, los próximos pasos implican la expansión continua del portafolio, la mejora de la documentación y la búsqueda de nuevas tecnologías y desafíos que estimulen la creatividad y el crecimiento profesional.\newpage

%\bibliographystyle{abntex2-alf} % o otro estilo, como plain, apalike...
%\bibliography{refer}

\begin{thebibliography}{}{
\fontsize{9pt}{10pt}\selectfont

\bibitem[wikipedia 2025]{whitepaper}
wikipedia: ``White paper''; disponible en: \url{https://en.wikipedia.org/wiki/White_paper}, consultado el: 21 de junio de 2025.

\bibitem[indeed 2025]{project}
indeed: ``How to Write a Project Overview (With Template and Example)''; disponible en: \url{https://ca.indeed.com/career-advice/career-development/how-to-write-project-overview}, consultado el: 13 de junio de 2025.

\bibitem[Wiegers 2012]{req_inverse}
Wiegers, K.: ``Requirements Management''; disponible en: \url{https://static1.squarespace.com/static/50c9c50fe4b0a97682fac903/t/50feb37ce4b000014e7f1191/1358869372660/Karl+Wiegers+Writing+High+Quality+Requirements-1.pdf}, consultado el: 20 de mayo de 2025.

\bibitem[geeksforgeeks 2025]{req_Func_NonFunc}
geeksforgeeks: ``Functional vs. Non Functional Requirements''; disponible en: \url{https://www.geeksforgeeks.org/functional-vs-non-functional-requirements/}, consultado el: 20 de mayo de 2025.

\bibitem[Scoreplan 2025]{SWOT}
Scoreplan: ``SWOT Analysis: What It Is and How to Use It to Grow Your Business''; disponible en: \url{https://scoreplan.com.br/swot-analysis/}, consultado el: 20 de mayo de 2025.

\bibitem[wikipedia 2025]{MIT}
wikipedia: ``MIT License''; disponible en: \url{https://en.wikipedia.org/wiki/MIT_License}, consultado el: 20 de mayo de 2025.

}
\end{thebibliography}




\end{document}
