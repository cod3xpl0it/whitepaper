\documentclass[10pt, a4paper, oneside]{article}
\usepackage[hidelinks]{hyperref}
\usepackage{jucs2e}
\usepackage{graphicx}
\usepackage{url}
\usepackage{ulem}
\usepackage{mathtools}
\usepackage{scalerel}
\usepackage{setspace}
\usepackage[strict]{changepage}
\usepackage{caption}
\usepackage[letterspace=-50]{microtype}
%\usepackage{fontspec}
\usepackage{afterpage}
\usepackage{ragged2e}
%\setmainfont{Times New Roman}
\usepackage[T1]{fontenc}
\usepackage[utf8]{inputenc}
\usepackage{times} % Oppure \usepackage{mathptmx}
\usepackage{titlesec}
\usepackage[italian]{babel}
\usepackage[alf]{abntex2cite} % oppure [author-year] per altro stile

\titleformat*{\section}{\Large\bfseries}
\titleformat*{\subsection}{\normalsize\bfseries}

\renewcommand{\baselinestretch}{0.9} 

\graphicspath{{./figures/}}

\usepackage[textwidth=8cm, margin=0cm, left=4.6cm, right=4.2cm, top=3.9cm, bottom=6.8cm, a4paper, headheight=0.5cm, headsep=0.5cm]{geometry}
\usepackage{fancyhdr}
\usepackage[format=plain, labelfont=it, textfont=it, justification=centering]{caption}
\usepackage{breakcites}
\usepackage{microtype}
 
\apptocmd{\frame}{}{\justifying}{}

\urlstyle{same}
\pagestyle{fancy}

\newcommand\jucs{Codexplo.it Project Whitepaper}
\newcommand\jucsvol{Version 0.01}
\newcommand\jucspages{pages: 1--7} % da regolare secondo il numero di pagine
\renewcommand\jucssubmitted{19 giugno 2025} % data attuale di compilazione
\newcommand\jucsaccepted{26 giugno 2025} % se non c’è accettazione formale, si può lasciare vuoto o scrivere "N/A"
\newcommand\jucsappeared{5 luglio 2025} % data di pubblicazione o ultima revisione
\newcommand\jucslicence{, Licenza: MIT License}
\newcommand\startingPage{1}
\setcounter{page}{\startingPage}

%Autore
\newcommand\paperauthor{{Vilela C.: }}

% Titolo
\newcommand\papertitle{Whitepaper \\ Progetto Codexplo.it}

% icone
\usepackage{fontawesome}
\usepackage{hyperref}

% Contenuto intestazione principale
\header{\paperauthor \papertitle}

\begin{document}

\title{{\fontsize{14pt}{14pt}\selectfont{\vspace*{-3mm}\papertitle\vspace*{-1mm}}}}

\author{%
  {\bfseries\fontsize{10pt}{12pt}\selectfont Carlos Vilela}\\
  {\fontsize{9pt}{12pt}\selectfont codexplo.it (Progetto Personale)}\\
  {\faGithub\ \href{https://github.com/cod3xpl0it}{github.com/cod3xpl0it}}
  {\faGlobe\ \href{https://codexplo.it}{https://codexplo.it}}
}

\label{first}
\maketitle

{\fontfamily{ptm}\selectfont
\begin{abstract}
{\fontsize{9pt}{9pt}\selectfont{\vspace*{-2mm}
Questo whitepaper presenta Codexplo.it, un progetto personale di ingegneria informatica focalizzato sullo sviluppo di soluzioni innovative software e hardware attraverso l’apprendimento pratico e la sperimentazione tecnica, con l’obiettivo di esplorare tecniche avanzate di programmazione, sistemi embedded e architetture di computer, finalizzato a migliorare le competenze professionali, stimolare la creatività e costruire applicazioni utili da condividere con la comunità open source, oltre a creare un portfolio diversificato che dimostri competenze tecniche applicate in contesti reali.}}
\end{abstract}}

{\fontfamily{ptm}\selectfont
\begin{keywords}
{\fontsize{9pt}{9pt}\selectfont{
ingegneria informatica, sistemi embedded, sviluppo software, programmazione avanzata, apprendimento pratico, progetti personali, sperimentazione tecnologica, codice aperto}}
\end{keywords}}

\section{Introduzione}

L’ingegneria informatica è in costante evoluzione, adottando nuove tecnologie e metodologie, e questo whitepaper presenta Codexplo.it, un progetto personale che esplora soluzioni innovative software e hardware attraverso sperimentazione pratica e apprendimento, con l’obiettivo di migliorare le competenze professionali, stimolare la creatività e contribuire con progetti utili alla comunità tecnologica. Inoltre, il documento descrive la motivazione, gli obiettivi, le metodologie e i risultati attesi del progetto.

\section{Motivazione}

Il progetto Codexplo.it nasce dal desiderio di consolidare le conoscenze acquisite durante gli studi accademici e le esperienze personali, trasformando questo apprendimento in qualcosa di pratico, creativo e visibile tramite un portfolio personale. Questo spazio ha la finalità di esporre progetti personali sviluppati con creatività, libertà e sperimentazione, esplorando diverse aree dell’ingegneria informatica, sia nel software, hardware o nella loro integrazione.\newline

L’idea principale è costruire continuamente competenze tecniche e personali, applicando i concetti appresi formalmente e incorporando intuizioni provenienti dalla vita quotidiana che, anche fuori dall’ambito tecnico, hanno contribuito al mio sviluppo personale e professionale.\newpage

\section{Definizioni del Progetto}

\begin{itemize}
  \item \textbf{Nome del progetto:} Codexplo.it  
  \item \textbf{Sito ufficiale:} \texttt{https://codexplo.it}  
  \item \textbf{Repository GitHub:} \texttt{https://github.com/cod3xpl0it}  
  \item \textbf{Origine del nome:}  
  Il nome “codexplo.it” deriva dalla combinazione tra “code” (codice) e “exploit” (sfruttamento), esprimendo lo scopo del progetto di investigare e sperimentare tecnologie software e hardware.\newline
  
  Il dominio “.it” è un \textit{ccTLD} (country code Top-Level Domain), che identifica l’Italia su internet, incorporato intenzionalmente per comporre il nome in modo creativo, rendendolo più corto, moderno e facile da ricordare.  
\end{itemize}

\section{Panoramica del Progetto}

Il progetto \textbf{Codexplo.it} è un’iniziativa personale volta a integrare e organizzare diverse esperienze e soluzioni tecniche sviluppate nei settori software e hardware, promuovendo l’apprendimento pratico e lo sviluppo continuo di competenze in ingegneria informatica, indirizzato ad appassionati, studenti e professionisti che valorizzano l’innovazione e l’applicazione reale della conoscenza tecnica, offrendo un portfolio accessibile e trasparente che illustra l’evoluzione delle competenze acquisite nel tempo, incentivando lo scambio di idee e la collaborazione nella comunità tecnologica.

\section{Obiettivi}

Il progetto \textbf{Codexplo.it} ha come obiettivo principale quello di servire come piattaforma personale di sperimentazione, apprendimento e dimostrazione di soluzioni tecnologiche, raccogliendo creazioni originali in un portfolio accessibile, organizzato e tecnicamente rilevante. Tra i suoi obiettivi specifici vi sono:

\begin{itemize}
  \item Applicare in modo pratico le conoscenze acquisite sia nella formazione accademica che nelle esperienze personali, promuovendo un apprendimento più profondo.
  \item Sviluppare e documentare progetti originali, consentendo una riflessione continua sui processi, decisioni tecniche e risultati raggiunti.
  \item Stimolare la creatività, il pensiero critico e l’autonomia, attraverso la libertà di esplorare diverse tecnologie, metodologie e linguaggi di programmazione.
  \item Costruire e migliorare competenze tecniche e interpersonali, con focus sulla crescita professionale e la creazione di un vantaggio competitivo.
  \item Creare un repository pubblico che possa ispirare, collaborare e contribuire alla comunità tecnologica.\newpage
\end{itemize}

\section{Metodologia}

La metodologia adottata nel progetto \textbf{Codexplo.it} si basa su un approccio esplorativo e pratico, in cui ogni progetto sviluppato funge da opportunità di apprendimento attivo e sviluppo di competenze. L’idea centrale è applicare le conoscenze acquisite, esplorare nuove tecnologie, proporre soluzioni creative e riflettere sui risultati, tutto ciò attraverso un processo continuo, iterativo e autonomo.

Il processo metodologico adottato segue una struttura flessibile, composta da fasi che si adattano in base alle sfide da affrontare:

\begin{itemize}
  \item \textbf{Identificazione del problema o sfida}: definizione dell’ambito, obiettivi e soluzioni potenziali.
  \item \textbf{Ricerca e fondamento}: raccolta di riferimenti tecnici, buone pratiche e tecnologie valide.
  \item \textbf{Sviluppo pratico}: implementazione della soluzione utilizzando strumenti e linguaggi adeguati.
  \item \textbf{Documentazione e riflessione}: registrazione del processo, decisioni prese e lezioni apprese.
  \item \textbf{Condivisione}: pubblicazione dei risultati nel portfolio, inclusi codice sorgente, documentazione, descrizioni e possibili miglioramenti futuri.
\end{itemize}

Ogni sfida inizia con un problema, seguito da ricerca, creazione della soluzione e registrazione dell’intero processo, come codici e apprendimento, condividendo la documentazione nel portfolio, promuovendo lo scambio di conoscenze e l’evoluzione continua.

\subsection{Storico}

L’idea alla base del progetto \textbf{Codexplo.it} è nata all’inizio di gennaio 2025 come iniziativa personale per consolidare l’apprendimento tecnico e la sperimentazione pratica; col tempo, il concetto è evoluto, dando origine a un progetto strutturato, con identità e scopo chiari.

\begin{table}[ht]
    \centering
    \setlength\tabcolsep{6pt}
    \begin{tabular}{ |l|l|p{6.8cm}| }
        \hline
        \bfseries Data & \bfseries Fase & \bfseries Descrizione \\
        \hline
        06/01/2025 & Ideazione & Inizio della riflessione sull’importanza di raccogliere progetti personali in un portfolio tecnico, enfatizzando l’apprendimento teorico e pratico \\
        \hline
        Feb–Apr/2025 & Concezione & Sviluppo dell’identità del progetto, definizione del nome, organizzazione della proposta e elaborazione dei piani strutturali iniziali \\
        \hline
        11/05/2025 & Registrazione & Acquisizione ufficiale del dominio \texttt{codexplo.it}, segnando l’inizio formale del progetto come piattaforma pubblica e accessibile \\
        \hline
        Mag/2025–Presente & Implementazione & Sviluppo di contenuti, organizzazione dei progetti precedenti, applicazione di metodologie pratiche e costruzione continua del portfolio \\
        \hline
    \end{tabular}
    \caption{\fontsize{10pt}{11pt}\selectfont{\itshape{Evoluzione del progetto Codexploit dalla concezione fino alla fase attuale di implementazione}}}
    \label{table:history}
\end{table}
\newpage

\section{Requisiti del Progetto}

Perché il progetto \textbf{Codexploit} si sviluppi in modo coerente e raggiunga i suoi obiettivi, è essenziale definire requisiti chiari che guidino le sue attività, organizzazione e risultati attesi, dividendo tali requisiti in funzionali, non funzionali e inversi, comprendendo sia l’esecuzione pratica che i limiti e le linee guida del progetto.

\subsection{Requisiti Funzionali}

I requisiti funzionali specificano le azioni e le consegne che il progetto deve fornire per adempiere allo scopo di raccogliere e presentare problematiche, favorire l’apprendimento e documentare i progressi tecnici.

\begin{itemize}
    \item Organizzare e presentare le problematiche in modo chiaro, accessibile e strutturato, facilitando il monitoraggio dello sviluppo tecnico.
    \item Documentare il processo di apprendimento attraverso sperimentazione pratica, test e validazione continua.
    \item Condividere risultati, codici e documentazione che possano contribuire alla comunità tecnologica.
    \item Aggiornare il portfolio con nuove problematiche e miglioramenti, riflettendo l’evoluzione delle competenze nel tempo.
    \item Mantenere una comunicazione trasparente sugli obiettivi, metodologie e progresso del progetto.
\end{itemize}

\subsection{Requisiti Non Funzionali}

I requisiti non funzionali stabiliscono qualità e vincoli che devono guidare il progetto per garantirne sostenibilità, rilevanza e impatto positivo.

\begin{itemize}
    \item Mantenere un approccio pratico e autodidatta, valorizzando la sperimentazione e l’apprendimento continuo.
    \item Assicurare coerenza e qualità nella documentazione e nel contenuto presentato.
    \item Garantire l’accessibilità del portfolio per diversi pubblici, rispettando standard di usabilità e chiarezza.
    \item Adottare pratiche etiche, rispettando diritti d’autore e privacy nelle informazioni condivise.
    \item Conservare la flessibilità per adattare e incorporare nuove tecnologie o aree di interesse.
\end{itemize}

\subsection{Requisiti Inversi}

I requisiti inversi definiscono limitazioni e comportamenti da evitare per preservare il focus, la qualità e l’integrità.

\begin{itemize}
    \item Evitare la dispersione del focus del progetto con temi o contenuti non correlati all’ingegneria informatica e allo sviluppo personale.
    \item Prevenire la pubblicazione di informazioni personali sensibili o dati che possano compromettere la privacy.
    \item Non sacrificare la qualità della documentazione a favore della quantità di progetti presentati.
    \item Evitare pratiche che compromettano originalità o etica, come plagio o uso non autorizzato di materiali protetti.
    \item Impedire approcci che limitino la crescita o restringano sperimentazione, innovazione e creatività.
\end{itemize}

\section{Tecnologie e Strumenti Utilizzati}

Le tecnologie utilizzate nei progetti che compongono \textbf{Codexplo.it} sono scelte in base agli obiettivi di ogni problematica, privilegiando strumenti gratuiti, aperti e senza restrizioni, per garantire facile accesso, incoraggiare la sperimentazione e supportare l’apprendimento, cercando sempre soluzioni flessibili, ben documentate e adeguate alle sfide presentate, mantenendo un equilibrio tra praticità, innovazione, sviluppo tecnico e creatività.

\section{Aspetti Legali e Licenze}

Il progetto \textbf{Codexplo.it} utilizza la licenza MIT, che consente l’uso, la modifica e la condivisione libera del contenuto, fornendo la libertà a tutti di contribuire e utilizzare il progetto.

\section{Analisi SWOT}

Segue un’analisi SWOT (Punti di Forza, Debolezze, Opportunità e Minacce) del progetto \textbf{Codexplo.it}, con focus sul suo sviluppo come iniziativa personale nel campo dell’ingegneria informatica.

\subsection*{Punti di Forza (Strengths)}
\begin{itemize}
  \item Progetto autonomo, con totale libertà creativa.
  \item Focus sull’apprendimento pratico e continuo.
  \item Documentazione organizzata e strutturata.
  \item Licenza MIT che permette collaborazione aperta.
  \item Integrazione tra software e hardware.
\end{itemize}

\subsection*{Debolezze (Weaknesses)}
\begin{itemize}
  \item Dipende esclusivamente da tempo e motivazione personale.
  \item Risorse limitate per progetti più complessi.
\end{itemize}

\subsection*{Opportunità (Opportunities)}
\begin{itemize}
  \item Coinvolgimento con la comunità open source.
  \item Possibilità di trasformare contenuti in materiale educativo.
  \item Potenziale per partnership e collaborazioni esterne.
  \item Pubblicazione su piattaforme tecniche come GitHub.
\end{itemize}

\subsection*{Minacce (Threats)}
\begin{itemize}
  \item Obsolescenza delle tecnologie utilizzate.
  \item Concorrenza con progetti più strutturati o finanziati.
  \item Esposizione di errori tecnici o vulnerabilità di sicurezza.
\end{itemize}

\section{Conclusioni e Prossimi Passi}

Questo whitepaper ha presentato lo scopo, gli obiettivi, la metodologia e la struttura del progetto \textbf{Codexploit}, dimostrando il suo potenziale come piattaforma per apprendimento pratico e sviluppo tecnico. I prossimi passi prevedono l’espansione continua del portfolio, il miglioramento della documentazione e la ricerca di nuove tecnologie e sfide che stimolino creatività e crescita professionale.\newpage

%\bibliographystyle{abntex2-alf} % o altro stile, come plain, apalike...
%\bibliography{refer}

\begin{thebibliography}{}{
\fontsize{9pt}{10pt}\selectfont

\bibitem[wikipedia 2025]{whitepaper}
wikipedia: ``White paper''; disponibile su: \url{https://en.wikipedia.org/wiki/White_paper}, consultato il: 21 giugno 2025.

\bibitem[indeed 2025]{project}
indeed: ``How to Write a Project Overview (With Template and Example)''; disponibile su: \url{https://ca.indeed.com/career-advice/career-development/how-to-write-project-overview}, consultato il: 13 giugno 2025.

\bibitem[Wiegers 2012]{req_inverse}
Wiegers, K.: ``Requirements Management''; disponibile su: \url{https://static1.squarespace.com/static/50c9c50fe4b0a97682fac903/t/50feb37ce4b000014e7f1191/1358869372660/Karl+Wiegers+Writing+High+Quality+Requirements-1.pdf}, consultato il: 20 maggio 2025.

\bibitem[geeksforgeeks 2025]{req_Func_NonFunc}
geeksforgeeks: ``Functional vs. Non Functional Requirements''; disponibile su: \url{https://www.geeksforgeeks.org/functional-vs-non-functional-requirements/}, consultato il: 20 maggio 2025.

\bibitem[Scoreplan 2025]{SWOT}
Scoreplan: ``SWOT Analysis: What It Is and How to Use It to Grow Your Business''; disponibile su: \url{https://scoreplan.com.br/swot-analysis/}, consultato il: 20 maggio 2025.

\bibitem[wikipedia 2025]{MIT}
wikipedia: ``MIT License''; disponibile su: \url{https://en.wikipedia.org/wiki/MIT_License}, consultato il: 20 maggio 2025.

}
\end{thebibliography}




\end{document}
