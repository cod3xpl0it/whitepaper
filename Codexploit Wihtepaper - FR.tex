\documentclass[10pt, a4paper, oneside]{article}
\usepackage[hidelinks]{hyperref}
\usepackage{jucs2e}
\usepackage{graphicx}
\usepackage{url}
\usepackage{ulem}
\usepackage{mathtools}
\usepackage{scalerel}
\usepackage{setspace}
\usepackage[strict]{changepage}
\usepackage{caption}
\usepackage[letterspace=-50]{microtype}
%\usepackage{fontspec}
\usepackage{afterpage}
\usepackage{ragged2e}
%\setmainfont{Times New Roman}
\usepackage[T1]{fontenc}
\usepackage[utf8]{inputenc}
\usepackage{times} % Ou \usepackage{mathptmx}
\usepackage{titlesec}
\usepackage[french]{babel}
\usepackage[alf]{abntex2cite} % ou [author-year] pour un autre style

\titleformat*{\section}{\Large\bfseries}
\titleformat*{\subsection}{\normalsize\bfseries}

\renewcommand{\baselinestretch}{0.9} 

\graphicspath{{./figures/}}

\usepackage[textwidth=8cm, margin=0cm, left=4.6cm, right=4.2cm, top=3.9cm, bottom=6.8cm, a4paper, headheight=0.5cm, headsep=0.5cm]{geometry}
\usepackage{fancyhdr}
\usepackage[format=plain, labelfont=it, textfont=it, justification=centering]{caption}
\usepackage{breakcites}
\usepackage{microtype}
 
\apptocmd{\frame}{}{\justifying}{}

\urlstyle{same}
\pagestyle{fancy}

\newcommand\jucs{Codexplo.it Project Whitepaper}
\newcommand\jucsvol{Version 0.01}
\newcommand\jucspages{pages: 1--7} % à ajuster selon le nombre de pages
\renewcommand\jucssubmitted{19 juin 2025} % date actuelle de compilation
\newcommand\jucsaccepted{26 juin 2025} % si aucune acceptation formelle, laissez vide ou mettez "N/A"
\newcommand\jucsappeared{5 juillet 2025} % date de publication ou dernière révision
\newcommand\jucslicence{, Licence : MIT License}
\newcommand\startingPage{1}
\setcounter{page}{\startingPage}

%Auteur
\newcommand\paperauthor{{Vilela C.: }}

% Titre
\newcommand\papertitle{Whitepaper \\ Projet Codexplo.it}

% icônes
\usepackage{fontawesome}
\usepackage{hyperref}

% Contenu de l’en-tête principal
\header{\paperauthor \papertitle}

\begin{document}

\title{{\fontsize{14pt}{14pt}\selectfont{\vspace*{-3mm}\papertitle\vspace*{-1mm}}}}

\author{%
  {\bfseries\fontsize{10pt}{12pt}\selectfont Carlos Vilela}\\
  {\fontsize{9pt}{12pt}\selectfont codexplo.it (Projet Personnel)}\\
  {\faGithub\ \href{https://github.com/cod3xpl0it}{github.com/cod3xpl0it}}
  {\faGlobe\ \href{https://codexplo.it}{https://codexplo.it}}
}

\label{first}
\maketitle

{\fontfamily{ptm}\selectfont
\begin{abstract}
{\fontsize{9pt}{9pt}\selectfont{\vspace*{-2mm}
Ce Whitepaper présente Codexplo.it, un projet personnel en ingénierie informatique axé sur le développement de solutions innovantes en logiciel et matériel à travers l’apprentissage pratique et l’expérimentation technique, visant à explorer des techniques avancées de programmation, systèmes embarqués et architectures informatiques, dans le but d’améliorer les compétences professionnelles, stimuler la créativité et construire des applications utiles pouvant être partagées avec la communauté open source, ainsi que de constituer un portfolio diversifié démontrant des compétences techniques appliquées dans des contextes réels.}}
\end{abstract}}

{\fontfamily{ptm}\selectfont
\begin{keywords}
{\fontsize{9pt}{9pt}\selectfont{
ingénierie informatique, systèmes embarqués, développement logiciel, programmation avancée, apprentissage pratique, projets personnels, expérimentation technologique, code ouvert}}
\end{keywords}}

\section{Introduction}

L’ingénierie informatique est en constante évolution, adoptant de nouvelles technologies et méthodologies, et ce Whitepaper présente Codexplo.it, un projet personnel qui explore des solutions innovantes en logiciel et matériel par l’expérimentation pratique et l’apprentissage, visant à améliorer les compétences professionnelles, stimuler la créativité et contribuer avec des projets utiles à la communauté technologique. De plus, le document décrit la motivation, les objectifs, les méthodologies et les résultats attendus du projet.

\section{Motivation}

Le projet Codexplo.it est né du désir de consolider les connaissances acquises au cours des études académiques et des expériences personnelles, transformant cet apprentissage en quelque chose de pratique, créatif et visible à travers un portfolio personnel. Cet espace a pour but d’exposer des projets personnels développés avec créativité, liberté et expérimentation, explorant différents domaines de l’ingénierie informatique, que ce soit en logiciel, matériel ou dans leur intégration.\newline

L’idée principale est de construire continuellement des compétences techniques et personnelles, en appliquant des concepts appris formellement et en incorporant des idées provenant du quotidien qui, même hors du domaine technique, ont contribué à mon développement personnel et professionnel.\newpage

\section{Définitions du Projet}

\begin{itemize}
  \item \textbf{Nom du projet :} Codexplo.it  
  \item \textbf{Site officiel :} \texttt{https://codexplo.it}  
  \item \textbf{Dépôt GitHub :} \texttt{https://github.com/cod3xpl0it}  
  \item \textbf{Origine du nom :}  
  Le nom “codexplo.it” résulte de la combinaison de “code” (code) et “exploit” (exploitation), exprimant le but du projet d’investiguer et expérimenter des technologies en logiciel et matériel.\newline
  
  Le domaine “.it” est un \textit{ccTLD} (country code Top-Level Domain), qui identifie l’Italie sur internet, incorporé intentionnellement pour composer le nom de manière créative, le rendant plus court, moderne et facile à mémoriser.  
\end{itemize}

\section{Vue d’ensemble du Projet}

Le projet \textbf{Codexplo.it} est une initiative personnelle visant à intégrer et organiser diverses expériences et solutions techniques développées dans les domaines du logiciel et du matériel, promouvant l’apprentissage pratique et le développement continu de compétences en ingénierie informatique, destiné aux passionnés, étudiants et professionnels qui valorisent l’innovation et l’application réelle des connaissances techniques, offrant un portfolio accessible et transparent illustrant l’évolution des compétences acquises au fil du temps, tout en encourageant l’échange d’idées et la collaboration au sein de la communauté technologique.

\section{Objectifs}

Le projet \textbf{Codexplo.it} a pour objectif principal de servir de plateforme personnelle d’expérimentation, d’apprentissage et de démonstration de solutions technologiques, rassemblant des créations originales dans un portfolio accessible, organisé et techniquement pertinent. Parmi ses objectifs spécifiques figurent :

\begin{itemize}
  \item Appliquer de manière pratique les connaissances acquises tant dans la formation académique que dans les expériences personnelles, favorisant un apprentissage plus approfondi.
  \item Développer et documenter des projets originaux, permettant une réflexion continue sur les processus, décisions techniques et résultats obtenus.
  \item Stimuler la créativité, la pensée critique et l’autonomie, à travers la liberté d’explorer diverses technologies, méthodologies et langages de programmation.
  \item Construire et améliorer des compétences techniques et interpersonnelles, avec un focus sur la croissance professionnelle et la création d’un avantage compétitif.
  \item Créer un dépôt public pouvant inspirer, collaborer et contribuer à la communauté technologique.\newpage
\end{itemize}

\section{Méthodologie}

La méthodologie adoptée dans le projet \textbf{Codexplo.it} repose sur une approche exploratoire et pratique, où chaque projet développé fonctionne comme une opportunité d’apprentissage actif et de développement de compétences. L’idée centrale est d’appliquer les connaissances acquises, d’explorer de nouvelles technologies, de proposer des solutions créatives et de réfléchir aux résultats, le tout à travers un processus continu, itératif et autonome.

Le processus méthodologique adopté suit une structure flexible, composée d’étapes qui s’adaptent selon les défis à relever :

\begin{itemize}
  \item \textbf{Identification du problème ou défi} : définition du périmètre, des objectifs et des solutions potentielles.
  \item \textbf{Recherche et fondation} : collecte de références techniques, bonnes pratiques et technologies viables.
  \item \textbf{Développement pratique} : implémentation de la solution en utilisant des outils et langages adaptés.
  \item \textbf{Documentation et réflexion} : enregistrement du processus, décisions prises et leçons apprises.
  \item \textbf{Partage} : publication des résultats dans le portfolio, incluant code source, documentation, descriptions et améliorations possibles futures.
\end{itemize}

Chaque défi commence par un problème, suivi d’une recherche, création de la solution et enregistrement de tout le processus, comme les codes et apprentissages, partageant la documentation dans le portfolio, favorisant l’échange de connaissances et l’évolution continue.

\subsection{Historique}

L’idée derrière le projet \textbf{Codexplo.it} est née début janvier 2025 comme initiative personnelle pour consolider l’apprentissage technique et l’expérimentation pratique ; avec le temps, le concept a évolué, donnant naissance à un projet structuré, avec une identité et un but clairs.

\begin{table}[ht]
    \centering
    \setlength\tabcolsep{6pt}
    \begin{tabular}{ |l|l|p{6.8cm}| }
        \hline
        \bfseries Date & \bfseries Phase & \bfseries Description \\
        \hline
        06/01/2025 & Idéation & Début de la réflexion sur l’importance de rassembler des projets personnels dans un portfolio technique, mettant l’accent sur l’apprentissage théorique et pratique \\
        \hline
        Fév–Avr/2025 & Conception & Développement de l’identité du projet, définition du nom, organisation de la proposition et élaboration des plans structurels initiaux \\
        \hline
        11/05/2025 & Enregistrement & Acquisition officielle du domaine \texttt{codexplo.it}, marquant le début formel du projet comme plateforme publique et accessible \\
        \hline
        Mai/2025–Présent & Implémentation & Développement de contenu, organisation des projets précédents, application de méthodologies pratiques et construction continue du portfolio \\
        \hline
    \end{tabular}
    \caption{\fontsize{10pt}{11pt}\selectfont{\itshape{Évolution du projet Codexploit de la conception à la phase actuelle d’implémentation}}}
    \label{table:history}
\end{table}
\newpage

\section{Exigences du Projet}

Pour que le projet \textbf{Codexploit} se développe de manière cohérente et atteigne ses objectifs, il est essentiel de définir des exigences claires qui guident ses activités, son organisation et les résultats attendus, ces exigences étant divisées en fonctionnelles, non fonctionnelles et inverses, couvrant à la fois l’exécution pratique et les limites et directives du projet.

\subsection{Exigences Fonctionnelles}

Les exigences fonctionnelles spécifient les actions et livrables que le projet doit fournir pour remplir son but de rassembler et présenter les problématiques, favoriser l’apprentissage et documenter les progrès techniques.

\begin{itemize}
    \item Organiser et présenter les problématiques de manière claire, accessible et structurée, facilitant le suivi du développement technique.
    \item Documenter le processus d’apprentissage par le biais de l’expérimentation pratique, des tests et de la validation continue.
    \item Partager les résultats, codes et documentation pouvant contribuer à la communauté technologique.
    \item Mettre à jour le portfolio avec de nouvelles problématiques et améliorations, reflétant l’évolution des compétences au fil du temps.
    \item Maintenir une communication transparente sur les objectifs, méthodologies et progrès du projet.
\end{itemize}

\subsection{Exigences Non Fonctionnelles}

Les exigences non fonctionnelles définissent les qualités et contraintes qui doivent guider le projet afin d’assurer sa durabilité, sa pertinence et son impact positif.

\begin{itemize}
    \item Maintenir une approche pratique et autodidacte, valorisant l’expérimentation et l’apprentissage continu.
    \item Assurer la cohérence et la qualité de la documentation et du contenu présenté.
    \item Garantir l’accessibilité du portfolio pour différents publics, respectant les normes d’utilisabilité et de clarté.
    \item Adopter des pratiques éthiques, respectant les droits d’auteur et la vie privée dans les informations partagées.
    \item Préserver la flexibilité pour adapter et incorporer de nouvelles technologies ou domaines d’intérêt.
\end{itemize}

\subsection{Exigences Inverses}

Les exigences inverses définissent des limitations et comportements à éviter dans le projet afin de préserver le focus, la qualité et l’intégrité.

\begin{itemize}
    \item Éviter la dispersion du focus du projet avec des thèmes ou contenus non liés à l’ingénierie informatique et au développement personnel.
    \item Prévenir la publication d’informations personnelles sensibles ou de données pouvant compromettre la vie privée.
    \item Ne pas sacrifier la qualité de la documentation au profit de la quantité de projets présentés.
    \item Éviter les pratiques compromettant l’originalité ou l’éthique, comme le plagiat ou l’utilisation non autorisée de matériaux protégés.
    \item Empêcher des approches limitant la croissance ou restreignant l’expérimentation, l’innovation et la créativité.
\end{itemize}

\section{Technologies et Outils Utilisés}

Les technologies utilisées dans les projets qui composent \textbf{Codexplo.it} sont choisies en fonction des objectifs de chaque problématique, privilégiant des outils gratuits, ouverts et sans restrictions, pour garantir un accès facile, encourager l’expérimentation et soutenir l’apprentissage, recherchant toujours des solutions flexibles, bien documentées et adaptées aux défis présentés, en maintenant un équilibre entre praticité, innovation, développement technique et créativité.

\section{Aspects Légaux et Licences}

Le projet \textbf{Codexplo.it} utilise la licence MIT, qui permet d’utiliser, modifier et partager librement le contenu, offrant la liberté à tous de contribuer et d’utiliser le projet.

\section{Analyse SWOT}

Voici une analyse SWOT (Forces, Faiblesses, Opportunités et Menaces) du projet \textbf{Codexplo.it}, centrée sur son développement comme initiative personnelle dans le domaine de l’ingénierie informatique.

\subsection*{Forces (Strengths)}
\begin{itemize}
  \item Projet autonome, avec une liberté créative totale.
  \item Focus sur l’apprentissage pratique et continu.
  \item Documentation organisée et structurée.
  \item Licence MIT permettant une collaboration ouverte.
  \item Intégration entre logiciel et matériel.
\end{itemize}

\subsection*{Faiblesses (Weaknesses)}
\begin{itemize}
  \item Dépend exclusivement du temps et de la motivation personnelle.
  \item Ressources limitées pour des projets plus complexes.
\end{itemize}

\subsection*{Opportunités (Opportunities)}
\begin{itemize}
  \item Engagement avec la communauté open source.
  \item Possibilité de transformer le contenu en matériel éducatif.
  \item Potentiel de partenariats et collaborations externes.
  \item Publication sur des plateformes techniques comme GitHub.
\end{itemize}

\subsection*{Menaces (Threats)}
\begin{itemize}
  \item Obsolescence des technologies utilisées.
  \item Concurrence avec des projets plus structurés ou financés.
  \item Exposition à des erreurs techniques ou des failles de sécurité.
\end{itemize}

\section{Conclusion et Prochaines Étapes}

Ce Whitepaper a présenté la portée, les objectifs, la méthodologie et la structure du projet \textbf{Codexploit}, démontrant son potentiel comme plateforme d’apprentissage pratique et de développement technique. Les prochaines étapes impliquent l’expansion continue du portfolio, l’amélioration de la documentation et la recherche de nouvelles technologies et défis stimulant la créativité et la croissance professionnelle.\newpage

%\bibliographystyle{abntex2-alf} % ou autre style, comme plain, apalike...
%\bibliography{refer}

\begin{thebibliography}{}{
\fontsize{9pt}{10pt}\selectfont

\bibitem[wikipedia 2025]{whitepaper}
wikipedia: ``White paper''; disponible sur : \url{https://en.wikipedia.org/wiki/White_paper}, consulté le : 21 juin 2025.

\bibitem[indeed 2025]{project}
indeed: ``How to Write a Project Overview (With Template and Example)''; disponible sur : \url{https://ca.indeed.com/career-advice/career-development/how-to-write-project-overview}, consulté le : 13 juin 2025.

\bibitem[Wiegers 2012]{req_inverse}
Wiegers, K.: ``Requirements Management''; disponible sur : \url{https://static1.squarespace.com/static/50c9c50fe4b0a97682fac903/t/50feb37ce4b000014e7f1191/1358869372660/Karl+Wiegers+Writing+High+Quality+Requirements-1.pdf}, consulté le : 20 mai 2025.

\bibitem[geeksforgeeks 2025]{req_Func_NonFunc}
geeksforgeeks: ``Functional vs. Non Functional Requirements''; disponible sur : \url{https://www.geeksforgeeks.org/functional-vs-non-functional-requirements/}, consulté le : 20 mai 2025.

\bibitem[Scoreplan 2025]{SWOT}
Scoreplan: ``SWOT Analysis: What It Is and How to Use It to Grow Your Business''; disponible sur : \url{https://scoreplan.com.br/swot-analysis/}, consulté le : 20 mai 2025.

\bibitem[wikipedia 2025]{MIT}
wikipedia: ``MIT License''; disponible sur : \url{https://en.wikipedia.org/wiki/MIT_License}, consulté le : 20 mai 2025.

}
\end{thebibliography}




\end{document}
