\documentclass[10pt, a4paper, oneside]{article}
\usepackage[hidelinks]{hyperref}
\usepackage{jucs2e}
\usepackage{graphicx}
\usepackage{url}
\usepackage{ulem}
\usepackage{mathtools}
\usepackage{scalerel}
\usepackage{setspace}
\usepackage[strict]{changepage}
\usepackage{caption}
\usepackage[letterspace=-50]{microtype}
%\usepackage{fontspec}
\usepackage{afterpage}
\usepackage{ragged2e}
%\setmainfont{Times New Roman}
\usepackage[T1]{fontenc}
\usepackage[utf8]{inputenc}
\usepackage{times} % Or \usepackage{mathptmx}
\usepackage{titlesec}
\usepackage[english]{babel}
\usepackage[alf]{abntex2cite} % or [author-year] for other style

\titleformat*{\section}{\Large\bfseries}
\titleformat*{\subsection}{\normalsize\bfseries}

\renewcommand{\baselinestretch}{0.9} 

\graphicspath{{./figures/}}

\usepackage[textwidth=8cm, margin=0cm, left=4.6cm, right=4.2cm, top=3.9cm, bottom=6.8cm, a4paper, headheight=0.5cm, headsep=0.5cm]{geometry}
\usepackage{fancyhdr}
\usepackage[format=plain, labelfont=it, textfont=it, justification=centering]{caption}
\usepackage{breakcites}
\usepackage{microtype}
 
\apptocmd{\frame}{}{\justifying}{}

\urlstyle{same}
\pagestyle{fancy}

\newcommand\jucs{Codexplo.it Project Whitepaper}
\newcommand\jucsvol{Version 0.01}
\newcommand\jucspages{pages: 1--7} % adjust according to the number of pages
\renewcommand\jucssubmitted{Jun 19, 2025} % current compilation date
\newcommand\jucsaccepted{Jun 26, 2025} % if there is no formal acceptance, you can leave empty or put "N/A"
\newcommand\jucsappeared{Jul 5, 2025} % publication date or last revision date
\newcommand\jucslicence{, License: MIT License}
\newcommand\startingPage{1}
\setcounter{page}{\startingPage}

%Author
\newcommand\paperauthor{{Vilela C.: }}

% Title
\newcommand\papertitle{Whitepaper \\ Codexplo.it Project}

% icon
\usepackage{fontawesome}
\usepackage{hyperref}

% Main header content
\header{\paperauthor \papertitle}

\begin{document}

\title{{\fontsize{14pt}{14pt}\selectfont{\vspace*{-3mm}\papertitle\vspace*{-1mm}}}}

\author{%
  {\bfseries\fontsize{10pt}{12pt}\selectfont Carlos Vilela}\\
  {\fontsize{9pt}{12pt}\selectfont codexplo.it (Personal Project)}\\
  {\faGithub\ \href{https://github.com/cod3xpl0it}{github.com/cod3xpl0it}}
  {\faGlobe\ \href{https://codexplo.it}{https://codexplo.it}}
}

\label{first}
\maketitle

{\fontfamily{ptm}\selectfont
\begin{abstract}
{\fontsize{9pt}{9pt}\selectfont{\vspace*{-2mm}
This whitepaper presents Codexplo.it, a personal computer engineering project focused on developing innovative software and hardware solutions through hands-on learning and technical experimentation, aiming to explore advanced programming techniques, embedded systems, and computer architectures, with the goal of enhancing professional skills, stimulating creativity, and building useful applications that can be shared with the open source community, as well as creating a diverse portfolio demonstrating technical competencies applied in real-world contexts.}}
\end{abstract}}

{\fontfamily{ptm}\selectfont
\begin{keywords}
{\fontsize{9pt}{9pt}\selectfont{
computer engineering, embedded systems, software development, advanced programming, hands-on learning, personal projects, technological experimentation, open source}}
\end{keywords}}

\section{Introduction}

Computer engineering is constantly evolving, adopting new technologies and methodologies, and this whitepaper introduces Codexplo.it, a personal project exploring innovative software and hardware solutions through practical experimentation and learning, aiming to improve professional skills, stimulate creativity, and contribute useful projects to the technological community. Additionally, the document describes the motivation, objectives, methodologies, and expected outcomes of the project.

\section{Motivation}

The Codexplo.it project was born from the desire to consolidate knowledge acquired throughout academic studies and personal experiences, transforming this learning into something practical, creative, and visible through a personal portfolio. This space aims to showcase personal projects developed with creativity, freedom, and experimentation, exploring different areas of computer engineering, whether in software, hardware, or their integration.\newline

The main idea is to continuously build technical and personal skills, applying formally learned concepts and incorporating insights from everyday life which, even outside the technical field, have contributed to my personal and professional development.\newpage

\section{Project Definitions}

\begin{itemize}
  \item \textbf{Project name:} Codexplo.it  
  \item \textbf{Official website:} \texttt{https://codexplo.it}  
  \item \textbf{GitHub repository:} \texttt{https://github.com/cod3xpl0it}  
  \item \textbf{Name origin:}  
  The name “codexplo.it” results from the combination of “code” and “exploit,” expressing the project’s purpose to investigate and experiment with software and hardware technologies.\newline
  
  The “.it” domain is a \textit{ccTLD} (country code Top-Level Domain) that identifies Italy on the internet, intentionally incorporated to compose the name creatively, making it shorter, modern, and easy to remember.  
\end{itemize}

\section{Project Overview}

The \textbf{Codexplo.it} project is a personal initiative aimed at integrating and organizing diverse technical experiences and solutions developed in software and hardware areas, promoting hands-on learning and continuous development of competencies in computer engineering, directed at enthusiasts, students, and professionals who value innovation and real application of technical knowledge, offering an accessible and transparent portfolio that illustrates the evolution of skills acquired over time, while encouraging idea exchange and collaboration within the tech community.

\section{Objectives}

The \textbf{Codexplo.it} project has as its main objective to serve as a personal platform for experimentation, learning, and demonstration of technological solutions, gathering original creations in an accessible, organized, and technically relevant portfolio. Its specific goals include:

\begin{itemize}
  \item Apply, in practice, knowledge acquired both in academic training and personal experiences, promoting deeper learning.
  \item Develop and document original projects, enabling continuous reflection on processes, technical decisions, and achieved results.
  \item Stimulate creativity, critical thinking, and autonomy through freedom to explore diverse technologies, methodologies, and programming languages.
  \item Build and improve technical and interpersonal skills, focusing on professional growth and creating a competitive edge.
  \item Create a public repository that can inspire, collaborate, and contribute to the technological community.\newpage
\end{itemize}

\section{Methodology}

The methodology adopted in the \textbf{Codexplo.it} project is based on an exploratory and practical approach, where each developed project acts as an opportunity for active learning and skills development. The central idea is to apply acquired knowledge, explore new technologies, propose creative solutions, and reflect on results, all through a continuous, iterative, and autonomous process.

The adopted methodological process follows a flexible structure, composed of stages adapted according to the challenges to be faced:

\begin{itemize}
  \item \textbf{Problem or challenge identification}: defining scope, objectives, and potential solutions.
  \item \textbf{Research and grounding}: gathering technical references, best practices, and viable technologies.
  \item \textbf{Practical development}: implementing the solution using appropriate tools and languages.
  \item \textbf{Documentation and reflection}: recording the process, decisions made, and lessons learned.
  \item \textbf{Sharing}: publishing results in the portfolio, including source code, documentation, descriptions, and possible future improvements.
\end{itemize}

Each challenge starts with a problem, followed by research, solution creation, and recording of the entire process, such as codes and learnings, sharing documentation in the portfolio, promoting knowledge exchange and continuous evolution.

\subsection{History}

The idea behind the \textbf{Codexplo.it} project arose in early January 2025 as a personal initiative to consolidate technical learning and practical experimentation; over time, the concept evolved, giving rise to a structured project with clear identity and purpose.

\begin{table}[ht]
    \centering
    \setlength\tabcolsep{6pt}
    \begin{tabular}{ |l|l|p{6.8cm}| }
        \hline
        \bfseries Date & \bfseries Phase & \bfseries Description \\
        \hline
        01/06/2025 & Ideation & Beginning reflection on the importance of gathering personal projects in a technical portfolio, emphasizing theoretical and practical learning \\
        \hline
        Feb–Apr/2025 & Conception & Project identity development, name definition, proposal organization, and initial structural plan creation \\
        \hline
        05/11/2025 & Registration & Official acquisition of the domain \texttt{codexplo.it}, marking the formal start of the project as a public and accessible platform \\
        \hline
        May/2025–Present & Implementation & Content development, organization of previous projects, application of practical methodologies, and continuous portfolio building \\
        \hline
    \end{tabular}
    \caption{\fontsize{10pt}{11pt}\selectfont{\itshape{Evolution of the Codexplo.it project from conception to current implementation phase}}}
    \label{table:history}
\end{table}
\newpage

\section{Project Requirements}

For the \textbf{Codexploit} project to develop consistently and achieve its objectives, it is essential to define clear requirements that guide its activities, organization, and expected results, these requirements being divided into functional, non-functional, and inverse, encompassing both practical execution and project limits and guidelines.

\subsection{Functional Requirements}

Functional requirements specify the actions and deliverables the project must provide to fulfill its purpose of gathering and presenting problems, fostering learning, and documenting technical progress.

\begin{itemize}
    \item Organize and present problems clearly, accessibly, and structured, facilitating the tracking of technical development.
    \item Document the learning process through practical experimentation, testing, and continuous validation.
    \item Share results, code, and documentation that can contribute to the technological community.
    \item Update the portfolio with new problems and improvements, reflecting the evolution of skills over time.
    \item Maintain transparent communication about the project’s objectives, methodologies, and progress.
\end{itemize}

\subsection{Non-Functional Requirements}

Non-functional requirements establish qualities and constraints to guide the project ensuring sustainability, relevance, and positive impact.

\begin{itemize}
    \item Maintain a practical and self-taught approach, valuing experimentation and continuous learning.
    \item Ensure consistency and quality in documentation and presented content.
    \item Guarantee portfolio accessibility for different audiences, respecting usability and clarity standards.
    \item Adopt ethical practices, respecting copyrights and privacy in shared information.
    \item Preserve flexibility to adapt and incorporate new technologies or areas of interest.
\end{itemize}

\subsection{Inverse Requirements}

Inverse requirements define limitations and behaviors to be avoided to preserve focus, quality, and integrity.

\begin{itemize}
    \item Avoid dispersion of the project focus with topics or content unrelated to computer engineering and personal development.
    \item Prevent publication of sensitive personal information or data that may compromise privacy.
    \item Do not sacrifice documentation quality in favor of quantity of presented projects.
    \item Avoid practices that compromise originality or ethics, such as plagiarism or unauthorized use of protected materials.
    \item Prevent approaches that restrict growth or limit experimentation, innovation, and creativity.
\end{itemize}

\section{Technologies and Tools Used}

The technologies used in projects composing \textbf{Codexplo.it} are chosen based on the objectives of each problem, prioritizing free, open, and unrestricted tools to ensure easy access, encourage experimentation, and support learning, always seeking flexible, well-documented solutions suitable for the presented challenges, maintaining a balance between practicality, innovation, technical development, and creativity.

\section{Legal Aspects and Licensing}

The \textbf{Codexplo.it} project uses the MIT license, which allows free use, modification, and sharing of content, providing freedom for everyone to contribute and use the project.

\section{SWOT Analysis}

Below is a SWOT analysis (Strengths, Weaknesses, Opportunities, and Threats) of the \textbf{Codexplo.it} project, focusing on its development as a personal initiative in the computer engineering field.

\subsection*{Strengths}
\begin{itemize}
  \item Autonomous project with total creative freedom.
  \item Focus on practical and continuous learning.
  \item Organized and structured documentation.
  \item MIT license allowing open collaboration.
  \item Integration between software and hardware.
\end{itemize}

\subsection*{Weaknesses}
\begin{itemize}
  \item Depends exclusively on personal time and motivation.
  \item Limited resources for more complex projects.
\end{itemize}

\subsection*{Opportunities}
\begin{itemize}
  \item Engagement with the open source community.
  \item Possibility to turn content into educational material.
  \item Potential for partnerships and external collaborations.
  \item Publication on technical platforms like GitHub.
\end{itemize}

\subsection*{Threats}
\begin{itemize}
  \item Obsolescence of used technologies.
  \item Competition with more structured or funded projects.
  \item Exposure of technical errors or security flaws.
\end{itemize}

\section{Conclusion and Next Steps}

This whitepaper presented the scope, objectives, methodology, and structure of the \textbf{Codexploit} project, demonstrating its potential as a platform for practical learning and technical development. The next steps involve continuous portfolio expansion, documentation improvement, and seeking new technologies and challenges that stimulate creativity and professional growth.\newpage

%\bibliographystyle{abntex2-alf} % or another style like plain, apalike...
%\bibliography{refer}

\begin{thebibliography}{}{
\fontsize{9pt}{10pt}\selectfont

\bibitem[wikipedia 2025]{whitepaper}
wikipedia: ``White paper''; available at: \url{https://en.wikipedia.org/wiki/White_paper}, accessed on: June 21, 2025.

\bibitem[indeed 2025]{project}
indeed: ``How to Write a Project Overview (With Template and Example)''; available at: \url{https://ca.indeed.com/career-advice/career-development/how-to-write-project-overview}, accessed on: June 13, 2025.

\bibitem[Wiegers 2012]{req_inverse}
Wiegers, K.: ``Requirements Management''; available at: \url{https://static1.squarespace.com/static/50c9c50fe4b0a97682fac903/t/50feb37ce4b000014e7f1191/1358869372660/Karl+Wiegers+Writing+High+Quality+Requirements-1.pdf}, accessed on: May 20, 2025.

\bibitem[geeksforgeeks 2025]{req_Func_NonFunc}
geeksforgeeks: ``Functional vs. Non Functional Requirements''; available at: \url{https://www.geeksforgeeks.org/functional-vs-non-functional-requirements/}, accessed on: May 20, 2025.

\bibitem[Scoreplan 2025]{SWOT}
Scoreplan: ``SWOT Analysis: What It Is and How to Use It to Grow Your Business''; available at: \url{https://scoreplan.com.br/swot-analysis/}, accessed on: May 20, 2025.

\bibitem[wikipedia 2025]{MIT}
wikipedia: ``MIT License''; available at: \url{https://en.wikipedia.org/wiki/MIT_License}, accessed on: May 20, 2025.

}
\end{thebibliography}


\end{document}
