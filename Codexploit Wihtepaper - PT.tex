\documentclass[10pt, a4paper, oneside]{article}
\usepackage[hidelinks]{hyperref}
\usepackage{jucs2e}
\usepackage{graphicx}
\usepackage{url}
\usepackage{ulem}
\usepackage{mathtools}
\usepackage{scalerel}
\usepackage{setspace}
\usepackage[strict]{changepage}
\usepackage{caption}
\usepackage[letterspace=-50]{microtype}
%\usepackage{fontspec}
\usepackage{afterpage}
\usepackage{ragged2e}
%\setmainfont{Times New Roman}
\usepackage[T1]{fontenc}
\usepackage[utf8]{inputenc}
\usepackage{times} % Ou \usepackage{mathptmx}
\usepackage{titlesec}
\usepackage[brazil]{babel}
\usepackage[alf]{abntex2cite} % ou [author-year] para outro estilo



\titleformat*{\section}{\Large\bfseries}
\titleformat*{\subsection}{\normalsize\bfseries}

\renewcommand{\baselinestretch}{0.9} 

\graphicspath{{./figures/}}

\usepackage[textwidth=8cm, margin=0cm, left=4.6cm, right=4.2cm, top=3.9cm, bottom=6.8cm, a4paper, headheight=0.5cm, headsep=0.5cm]{geometry}
\usepackage{fancyhdr}
\usepackage[format=plain, labelfont=it, textfont=it, justification=centering]{caption}
\usepackage{breakcites}
\usepackage{microtype}
 
\apptocmd{\frame}{}{\justifying}{}


\urlstyle{same}
\pagestyle{fancy}

\newcommand\jucs{Codexplo.it Project Whitepaper}
\newcommand\jucsvol{Version 0.01}
\newcommand\jucspages{pages: 1--7} % adjust according to the number of pages
\renewcommand\jucssubmitted{Jun 19, 2025} % current compilation date
\newcommand\jucsaccepted{Jun 26, 2025} % if there is no formal acceptance, you can leave empty or put "N/A"
\newcommand\jucsappeared{Jul 5, 2025} % publication date or last revision date
\newcommand\jucslicence{, License: MIT License}
\newcommand\startingPage{1}
\setcounter{page}{\startingPage}


%Author
\newcommand\paperauthor{{Vilela C.: }}

% Title
\newcommand\papertitle{Whitepaper \\ Codexplo.it Project}

% icon
\usepackage{fontawesome}
\usepackage{hyperref}


% Main header content
\header{\paperauthor \papertitle}

\begin{document}

\title{{\fontsize{14pt}{14pt}\selectfont{\vspace*{-3mm}\papertitle\vspace*{-1mm}}}}

\author{%
  {\bfseries\fontsize{10pt}{12pt}\selectfont Carlos Vilela}\\
  {\fontsize{9pt}{12pt}\selectfont codexplo.it (Personal Project)}\\
  {\faGithub\ \href{https://github.com/cod3xpl0it}{github.com/cod3xpl0it}}
  {\faGlobe\ \href{https://codexplo.it}{https://codexplo.it}}
}


\label{first}
\maketitle

{\fontfamily{ptm}\selectfont
\begin{abstract}
{\fontsize{9pt}{9pt}\selectfont{\vspace*{-2mm}
Este whitepaper apresenta o Codexplo.it, um projeto pessoal de engenharia de computação focado no desenvolvimento de soluções inovadoras de software e hardware por meio do aprendizado prático e da experimentação técnica, com o objetivo de explorar técnicas avançadas de programação, sistemas embarcados e arquiteturas de computadores, visando aprimorar habilidades profissionais, estimular a criatividade e construir aplicações úteis que possam ser compartilhadas com a comunidade open source, além de construir um portfólio diversificado que demonstre competências técnicas aplicadas em contextos reais.}}
\end{abstract}}

{\fontfamily{ptm}\selectfont
\begin{keywords}
{\fontsize{9pt}{9pt}\selectfont{
engenharia de computação, sistemas embarcados, desenvolvimento de software, programação avançada, aprendizado prático, projetos pessoais, experimentação tecnológica, código aberto}}
\end{keywords}}




% ACM Categories explanation:
% C.3 — Special-Purpose and Application-Based Systems:
%   Covers specialized systems, including embedded systems and
%   specific applications, relevant to computer engineering and
%   integrated hardware/software development.
%
% D.2.5 — Testing and Debugging:
%   Refers to software testing and debugging techniques essential
%   for ensuring quality and correct operation of projects.
%
% D.3.3 — Language Constructs and Features:
%   Involves programming language features and constructs, fundamental
%   for coding and experimentation in your projects.
%
% I.2.6 — Learning:
%   Related to learning methods and training, including machine learning,
%   applied here to practical learning and technical experimentation.
%
% K.6.1 — Project and People Management:
%   Covers project and personnel management, reflecting organization
%   and planning aspects of personal project Codexploit.

%{\fontfamily{ptm}\selectfont
%\begin{category}
%{\fontsize{9pt}{9pt}\selectfont{
%C.3, D.2.5, D.3.3, I.2.6, K.6.1}}
%\end{category}}


%{\fontfamily{ptm}\selectfont
%\begin{doi}
%{\fontsize{9pt}{9pt}\selectfont{
%N/A}}
%\end{doi}}

\section{Introdução}

A engenharia de computação está em constante evolução, adotando novas tecnologias e metodologias, e este whitepaper apresenta o Codexplo.it, um projeto pessoal que explora soluções inovadoras de software e hardware por meio de experimentação prática e aprendizado, visando aprimorar habilidades profissionais, estimular a criatividade e contribuir com projetos úteis para a comunidade tecnológica. Além disso, o documento descreve a motivação, os objetivos, as metodologias e os resultados esperados do projeto.

\section{Motivação}

O projeto Codexplo.it nasceu do desejo de consolidar o conhecimento adquirido ao longo dos estudos acadêmicos e das experiências pessoais, transformando esse aprendizado em algo prático, criativo e visível por meio de um portfólio pessoal. Esse espaço tem a finalidade de expor projetos pessoais desenvolvidos com criatividade, liberdade e experimentação, explorando diferentes áreas da engenharia de computação, seja em software, hardware ou na integração entre ambos.\newline

A ideia principal é construir continuamente habilidades técnicas e pessoais, aplicando conceitos aprendidos formalmente e incorporando insights provenientes do cotidiano que, mesmo fora da área técnica, contribuíram para o meu desenvolvimento pessoal e profissional.\newpage

\section{Definições do Projeto}

\begin{itemize}
  \item \textbf{Nome do projeto:} Codexplo.it  
  \item \textbf{Site oficial:} \texttt{https://codexplo.it}  
  \item \textbf{Repositório GitHub:} \texttt{https://github.com/cod3xpl0it}  
  \item \textbf{Origem do nome:}  
  O nome “codexplo.it” resulta da combinação entre “code” (código) e “exploit” (exploração), expressando o propósito do projeto de investigar e experimentar tecnologias em software e hardware.\newline
  
  O domínio “.it” é um \textit{ccTLD} (country code Top-Level Domain), que identifica a Itália na internet, sendo incorporado intencionalmente para compor o nome de forma criativa, tornando-o mais curto, moderno e fácil de memorizar.  
\end{itemize}

\section{Visão Geral do Projeto}

O projeto \textbf{Codexplo.it} é uma iniciativa pessoal voltada para integrar e organizar diversas experiências e soluções técnicas desenvolvidas nas áreas de software e hardware, promovendo o aprendizado prático e o desenvolvimento contínuo de competências em engenharia de computação, direcionado a entusiastas, estudantes e profissionais que valorizam a inovação e a aplicação real do conhecimento técnico, oferecendo um portfólio acessível e transparente que ilustra a evolução das habilidades adquiridas ao longo do tempo, ao mesmo tempo em que incentiva o intercâmbio de ideias e a colaboração dentro da comunidade tecnológica.

\section{Objetivos}

O projeto \textbf{Codexplo.it} tem como objetivo principal servir como uma plataforma pessoal de experimentação, aprendizado e demonstração de soluções tecnológicas, reunindo criações originais em um portfólio acessível, organizado e tecnicamente relevante. Entre seus objetivos específicos estão:

\begin{itemize}
  \item Aplicar, de forma prática, o conhecimento adquirido tanto na formação acadêmica quanto nas experiências pessoais, promovendo um aprendizado mais profundo.
  \item Desenvolver e documentar projetos originais, possibilitando reflexão contínua sobre processos, decisões técnicas e resultados alcançados.
  \item Estimular a criatividade, o pensamento crítico e a autonomia, através da liberdade para explorar diversas tecnologias, metodologias e linguagens de programação.
  \item Construir e aprimorar habilidades técnicas e interpessoais, com foco no crescimento profissional e na criação de um diferencial competitivo.
  \item Criar um repositório público que possa inspirar, colaborar e contribuir com a comunidade tecnológica.\newpage
\end{itemize}




\section{Metodologia}

A metodologia adotada no projeto \textbf{Codexplo.it} baseia-se em uma abordagem exploratória e prática, onde cada projeto desenvolvido funciona como uma oportunidade de aprendizado ativo e desenvolvimento de habilidades. A ideia central é aplicar os conhecimentos adquiridos, explorar novas tecnologias, propor soluções criativas e refletir sobre os resultados, tudo isso por meio de um processo contínuo, iterativo e autônomo.

O processo metodológico adotado segue uma estrutura flexível, composta por etapas que se adaptam conforme os desafios a serem enfrentados:

\begin{itemize}
  \item \textbf{Identificação da problemática ou desafio}: definição do escopo, objetivos e soluções potenciais.
  \item \textbf{Pesquisa e fundamentação}: coleta de referências técnicas, boas práticas e tecnologias viáveis.
  \item \textbf{Desenvolvimento prático}: implementação da solução utilizando ferramentas e linguagens adequadas.
  \item \textbf{Documentação e reflexão}: registro do processo, decisões tomadas e lições aprendidas.
  \item \textbf{Compartilhamento}: publicação dos resultados no portfólio, incluindo código-fonte, documentação, descrições e possíveis melhorias futuras.
\end{itemize}

Cada desafio começa com uma problemática, seguida de pesquisa, criação da solução e registro de todo o processo, como códigos e aprendizados, compartilhando a documentação no portfólio, promovendo troca de conhecimento e evolução contínua.

\subsection{Histórico}

A ideia por trás do projeto \textbf{Codexplo.it} surgiu no início de Janeiro do ano de 2025 como uma iniciativa pessoal para consolidar o aprendizado técnico e a experimentação prática, com o tempo, o conceito evoluiu, dando origem a um projeto estruturado, com identidade e propósito claros.

\begin{table}[ht]
    \centering
    \setlength\tabcolsep{6pt}
    \begin{tabular}{ |l|l|p{6.8cm}| }
        \hline
        \bfseries Data & \bfseries Fase & \bfseries Descrição \\
        \hline
        06/01/2025 & Ideação & Início da reflexão sobre a importância de reunir projetos pessoais em um portfólio técnico, enfatizando o aprendizado teórico e prático \\
        \hline
        Fev–Abr/2025 & Concepção & Desenvolvimento da identidade do projeto, definição do nome, organização da proposta e elaboração dos planos estruturais iniciais \\
        \hline
        11/05/2025 & Registro & Aquisição oficial do domínio \texttt{codexplo.it}, marcando o início formal do projeto como uma plataforma pública e acessível \\
        \hline
        Mai/2025–Presente & Implementação & Desenvolvimento de conteúdo, organização dos projetos anteriores, aplicação de metodologias práticas e construção contínua do portfólio \\
        \hline
    \end{tabular}
    \caption{\fontsize{10pt}{11pt}\selectfont{\itshape{Evolução do projeto Codexploit desde a concepção até a fase atual de implementação}}}
    \label{table:history}
\end{table}
\newpage



\section{Requisitos do Projeto}

Para que o projeto \textbf{Codexploit} se desenvolva de forma consistente e alcance seus objetivos, é essencial definir requisitos claros que orientem suas atividades, organização e resultados esperados, sendo esses requisitos divididos em funcionais, não funcionais e inversos, abrangendo tanto a execução prática quanto os limites e diretrizes do projeto.

\subsection{Requisitos Funcionais}

Os requisitos funcionais especificam as ações e entregas que o projeto deve proporcionar para cumprir o propósito de reunir e apresentar as problemáticas, fomentar o aprendizado e documentar o progresso técnico.

\begin{itemize}
    \item Organizar e apresentar as problemáticas de forma clara, acessível e estruturada, facilitando o acompanhamento do desenvolvimento técnico.
    \item Documentar o processo de aprendizagem por meio de experimentação prática, testes e validação contínua.
    \item Compartilhar resultados, códigos e documentação que possam contribuir com a comunidade tecnológica.
    \item Atualizar o portfólio com novas problemáticas e melhorias, refletindo a evolução das habilidades ao longo do tempo.
    \item Manter uma comunicação transparente sobre os objetivos, metodologias e progresso do projeto.
\end{itemize}

\subsection{Requisitos Não Funcionais}

Os requisitos não funcionais estabelecem qualidades e restrições que devem guiar o projeto para garantir sua sustentabilidade, relevância e impacto positivo.

\begin{itemize}
    \item Manter uma abordagem prática e autodidata, valorizando a experimentação e o aprendizado contínuo.
    \item Assegurar consistência e qualidade na documentação e no conteúdo apresentado.
    \item Garantir acessibilidade do portfólio para diferentes públicos, respeitando padrões de usabilidade e clareza.
    \item Adotar práticas éticas, respeitando direitos autorais e privacidade nas informações compartilhadas.
    \item Preservar a flexibilidade para adaptar e incorporar novas tecnologias ou áreas de interesse.
\end{itemize}

\subsection{Requisitos Inversos}

Os requisitos inversos definem limitações e comportamentos que devem ser evitados no projeto para preservar o foco, qualidade e integridade.

\begin{itemize}
    \item Evitar dispersão do foco do projeto com temas ou conteúdos não relacionados à engenharia de computação e desenvolvimento pessoal.
    \item Prevenir a publicação de informações pessoais sensíveis ou dados que possam comprometer a privacidade.
    \item Não sacrificar a qualidade da documentação em favor da quantidade de projetos apresentados.
    \item Evitar práticas que comprometam a originalidade ou a ética, como plágio ou uso não autorizado de materiais protegidos.
    \item Impedir abordagens que restrinjam o crescimento ou limitem a experimentação, a inovação e criatividade.
\end{itemize}

\section{Tecnologias e Ferramentas Utilizadas}

As tecnologias utilizadas nos projetos que compõem o \textbf{Codexplo.it} são escolhidas com base nos objetivos de cada problemática, priorizando ferramentas gratuitas, abertas e sem restrições, para garantir fácil acesso, incentivar a experimentação e apoiar o aprendizado, sempre buscando soluções flexíveis, bem documentadas e adequadas aos desafios apresentados, mantendo um equilíbrio entre praticidade, inovação, desenvolvimento técnico e criatividade.



\section{Aspectos Legais e Licenciamento}

O projeto \textbf{Codexplo.it} usa a licença MIT, que permite usar, modificar e compartilhar o conteúdo livremente, proporcionando a liberdade para todos contribuírem e usarem o projeto.


\section{Análise SWOT}

A seguir é apresentada uma análise SWOT (Forças, Fraquezas, Oportunidades e Ameaças) do projeto \textbf{Codexplo.it}, com foco em seu desenvolvimento como iniciativa pessoal na área de engenharia de computação.

\subsection*{Forças (Strengths)}
\begin{itemize}
  \item Projeto autônomo, com total liberdade criativa.
  \item Foco em aprendizado prático e contínuo.
  \item Documentação organizada e estruturada.
  \item Licença MIT que permite colaboração aberta.
  \item Integração entre software e hardware.
\end{itemize}

\subsection*{Fraquezas (Weaknesses)}
\begin{itemize}
  \item Depende exclusivamente do tempo e motivação pessoal.
  \item Recursos limitados para projetos mais complexos.
\end{itemize}

\subsection*{Oportunidades (Opportunities)}
\begin{itemize}
  \item Engajamento com a comunidade open source.
  \item Possibilidade de transformar conteúdo em material educativo.
  \item Potencial para parcerias e colaborações externas.
  \item Publicação em plataformas técnicas como GitHub.
\end{itemize}

\subsection*{Ameaças (Threats)}
\begin{itemize}
  \item Obsolescência de tecnologias usadas.
  \item Concorrência com projetos mais estruturados ou financiados.
  \item Exposição de erros técnicos ou falhas de segurança.
\end{itemize}


\section{Conclusão e Próximos Passos}

Este whitepaper apresentou o escopo, objetivos, metodologia e estrutura do projeto \textbf{Codexploit}, demonstrando seu potencial como plataforma para aprendizado prático e desenvolvimento técnico. Os próximos passos envolvem a expansão contínua do portfólio, aprimoramento da documentação e a busca por novas tecnologias e desafios que estimulem a criatividade e o crescimento profissional.\newpage





%\bibliographystyle{abntex2-alf} % ou outro estilo, como plain, apalike...
%\bibliography{refer}


\begin{thebibliography}{}{
\fontsize{9pt}{10pt}\selectfont


\bibitem[wikipedia 2025]{whitepaper}
wikipedia: ``White paper''; disponível em: \url{https://en.wikipedia.org/wiki/White_paper}, acesso em: 21 jun. 2025.

\bibitem[indeed 2025]{project}
indeed: ``How to Write a Project Overview (With Template and Example)''; disponível em: \url{https://ca.indeed.com/career-advice/career-development/how-to-write-project-overview}, acesso em: 13 jun. 2025.

\bibitem[Wiegers 2012]{req_inverse}
Wiegers, K.: ``Requirements Management''; disponível em: \url{https://static1.squarespace.com/static/50c9c50fe4b0a97682fac903/t/50feb37ce4b000014e7f1191/1358869372660/Karl+Wiegers+Writing+High+Quality+Requirements-1.pdf}, acesso em: 20 mai. 2025.

\bibitem[geeksforgeeks 2025]{req_Func_NonFunc}
geeksforgeeks: ``Functional vs. Non Functional Requirements''; disponível em: \url{https://www.geeksforgeeks.org/functional-vs-non-functional-requirements/}, acesso em: 20 mai. 2025.

\bibitem[Scoreplan 2025]{SWOT}
Scoreplan: ``SWOT Analysis: What It Is and How to Use It to Grow Your Business''; disponível em: \url{https://scoreplan.com.br/swot-analysis/}, acesso em: 20 mai. 2025.

\bibitem[wikipedia 2025]{MIT}
wikipedia: ``MIT License''; disponível em: \url{https://en.wikipedia.org/wiki/MIT_License}, acesso em: 20 mai. 2025.

}\end{thebibliography}




\end{document}
